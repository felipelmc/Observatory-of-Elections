\section{A teoria da escolha racional (II): voto retrospectivo}

FIGUEIREDO, M. A decisão do voto, Introdução. \textit{Revista Compolítica}, Rio de Janeiro, v. 1,
n. 4, 2014.

\subsection{Fichamento}

\noindent \textit{Este é o poder do voto de um cidadão. Isto equivale a dizer que, se um indivíduo deixar de votar, sua ausência virtualmente não altera o resultado. Da mesma forma, segundo esta concepção, se um indivíduo votar em um candidato ou em outro, sua escolha não mudará em nada a classificação final de candidatos.} \cite[~p. 209]{marcus_figueiredo} \\

\noindent \textit{Por que as pessoas participam? Obter uma resposta a esta pergunta, compreender as razões que levam milhões a participar, é tornar inteligível a mensagem política transmitida através do voto popular. Este é o objetivo do livro: encontrar uma teoria que resolva o paradoxo da participação}. \cite[~p. 210]{marcus_figueiredo} \\

\noindent \textit{O voto apurado tem dois significados. Por um lado, ele é uma unidade que entrará na contabilidade total destinado a um candidato ou a um partido que, por meio de uma regra, se traduz em uma cadeira no Parlamento, ou no direito de alguém ser empossado em uma governadoria. Por outro lado, esse mesmo voto traz embutida uma declaração de vontade, de aspiração ou desejo de ver realizar-se alguma coisa.} \cite[~p. 211]{marcus_figueiredo} \\

\noindent \textit{Os diversos modelos de explicação deste fenômeno competem entre si exatamente ao tentar reconstruir o processo social que levou a um dado resultado eleitoral, e tentar também explicar por que ocorreu exatamente uma dada distribuição das vontades políticas e não outra.} \cite[~p. 211]{marcus_figueiredo} \\

\noindent \textit{Esta proposição definitória contempla os ingredientes relevantes que compõem o processo decisório dos eleitores: as propensões e as motivações individuais para a ação política, socialmente condicionadas. As divergências teóricas e epistemológicas entre os diversos modelos explicativos, como veremos, estão na identificação das origens das propensões e das motivações para a ação política.} \cite[~p. 213]{marcus_figueiredo} \\

\noindent \textit{[...] tentarei demonstrar que a decisão individual de abster-se ou participar como eleitor tem um fundamento racional, dissolvendo, com isto, o paradoxo da participação.} \cite[~p. 214]{marcus_figueiredo} \\

\noindent \textit{Para estabelecer o meu objetivo, devo adiantar que a solução da situação paradoxal do eleitor está em demonstrar que o poder discricionário do voto de um indivíduo, examinado dentro da dinâmica do processo eleitoral, transforma-se da ordem de $\frac{1}{N}$ para a ordem de $\frac{n+1}{N}$. Nesse sentido, é suficiente demonstrar porque é racional para o enésimo mais 1 eleitor participar, para que a decisão de participar dos restantes enésimos eleitores, por analogia, seja também racional.} \cite[~p. 215]{marcus_figueiredo}

\subsection{Notas de aula}

O texto fundador é \textit{Economic Retrospective Voting in Amarican National Elections: A Micro-Analysis} (1978), de Morris P. Fiorina (1946-). O argumento não é necessariamente de que a economia é o único fator na análise retrospectiva do voto, mas definitivamente é um dos mais importantes. Normalmente, a teoria fica famosa por conta da análise da economia: ``It's the economy, stupid!". 

\begin{description}
    \item [Hipóteses alternativas:] \textit{quanto mais longe do centro ideológico estiver um presidente, menos a economia tende a afetar o voto dos eleitores e o resultado eleitoral}. O eleitor pode olhar em retrospectiva e dizer que, de fato, economia talvez não seja uma forma de avaliar o governo. O que foi feito em relação à agenda prometida na campanha?
\end{description}

\noindent Existe uma relação inversa entre extremismo ideológico e programas políticos baseados em questões consensuais ou \textit{valence issues}, como crescimento econômico, combate à corrupção, investimentos em saúde e educação etc. \\

\noindent Enquanto candidatos centristas tendem a propor agendas de governo que enfatizam questões sobre as quais não há divergência entre a população, candidatos mais à esquerda e à direita tendem a enfatizar suas posições em questões menos consensuais, ou \textit{positional issues} (Curini, 2015). \\

\noindent Consequentemente, os parâmetros que os eleitores utilizam para avaliar se o governo atendeu às suas expectativas variam de acordo com o perfil ideológico do presidente. \\

\noindent \textbf{Argumento do Marcus:} Ficar em casa é mais racional do que ir votar. Afinal, seu voto não vale nada. 

\subsection{Fórum}

\begin{description}
    \item [Pergunta:] Que teoria explica melhor a vitória de Bolsonaro em 2018? Para responder, tenha em mente o que te parece ter sido o elemento mais decisivo para a maioria do eleitorado brasileiro optar pelo candidato do PSL e mobilize elementos da literatura sobre a(s) escola(s) de comportamento eleitoral que te pareça(m) mais convincente(s). Lembre-se de escolher apenas uma escola de pensamento (no máximo, faça referência a uma segunda escola para argumentar por oposição a ela).
\end{description}
