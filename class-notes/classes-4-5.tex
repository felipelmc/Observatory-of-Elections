\section{Os determinantes psicossociais do voto}

CAMPBELL, A. et al. Theoretical orientation. In: \_\_\_\_\_\_. \textit{The American voter}. Chicago, University of Chicago Press, 1960, p. 19-37.

\subsection{Fichamento}

Desde a primeira página do texto já faz referência a uma série de pesquisadores que não são do campo das ciências sociais stricto sensu. Em especial, fala-se em pesquisadores da estatística --- ``\textit{[...] coders, IBM operators, and statistical clerks who are inevitably involved in research operations of this magnitude.}" \cite{campbell}. De fato, é a valorização de pesquisas de opinião no âmbito nacional. \\

\noindent \textit{It has been unsatisfactory, however, to leave these two approaches as independent and competing bodies of theory. They are addressed to the same reality, and conflict between them is hardly a matter of contradictory findings. Rather, they are attacking the problem at different levels, and consequently in different languages. Each approach has had its characteristic strengths and shortcomings. To the degree that these strengths are complementary, the advantages of each should be preserved in a broader framework of theory.} \cite[~p. 18]{campbell}. Se refere às correntes sociológica e psicológica da ciência política. \\

\noindent \textit{If we are interested in voting behavior, it is likely that we wish to account for variation in at least two classes of events. We want to predict whether a given individual is going to vote, and which candidate hi will choose. Although these are pleasantly simple dependent variables, it is clear that they represent extremely complex behavior; no single-factor theory is likely to tell us much about them.} \cite[~p. 19]{campbell}. \\

\noindent \textit{Understanding is forced to range more widely than is prediction. At the level of prediction, once we have found a variable that forecasts our chosen event, we rest content. To find another, which in turn predicts the event, or which intervenes between it and the dependent event, is superfluous. Yet for the purpose of understanding, such additional factors are invaluable, although they do not improve out prediction of the final event materially. They do enhance our grasp of the total situation and the full range of conditions that operate within it.} \cite[~p. 20]{campbell}. 

\begin{description}
    \item [Causality: ] \textit{uniformities of sequence observed in time past, which may be expected in the absense of exogenous factors to hold in the future.} \cite[~p. 21]{campbell}
\end{description}

\noindent \textit{We cannot afford to build an explanatory model that treats each case as distinctive phenomenon, with unique mechanisms at work. A systematic theory must be able to accept a set of data pertaining to any individual case and provide an ultimate prediction behavior.} \cite[~p. 23]{campbell}

\begin{description}
    \item [Funnel of causality: ] \textit{The funnel shape is a logical product of the explanatory task chosen. Most of the complex events in the funnel occur as a result of multiple prior causes. Each such event is, in its turn, responsible for multiple effects as well, but out focus of interest narrows as we approach the dependent behavior. We progressively eliminate those effects that do not continue to have relevance for the political act. Since we are forced to take all partial causes as relevant at any juncture, relevant effects are therefore many fewer in number than relevant causes. The result is a convergence effect.} \cite[~p. 24]{campbell}

    \item [Exogenous factors: ] \textit{They include all those conditions that are so remote in nature from the content interest of the investigator that their inclusion in a system of variables, even if possible, would be undesirable.} \cite[~p. 25-26]{campbell}

    \item [Personal conditions: ] \textit{those events or states within the funnel of which the individual is aware, although he need not conceptualize them as the investigator does.} \cite[~p. 27]{campbell}
\end{description}

\noindent \textit{We assume that most events or conditions that bear directly upon behavior are perceived in some form or other by the individual prior to the determined behavior, and that \textbf{much of behavior consists of reactions to these perceptions.}} \cite[~p. 27]{campbell} \\

\noindent \textit{Out theoretical superstructure immediately poses several such questions. What cross sections in time deserve our most immediate attention? What shall we exclude as exogenous factors? How far back shall we attempt to explore in the infinite regress of antecedent factors?} \cite[~p. 33]{campbell} \\

\noindent \textit{[...] the attitudinal approach entails some liabilities as well. Measurement close to the behavior runs the rusk of including values that are determined by the event we are trying to predict --- that is, the vote decision. To the degree that this occurs, some elements of a system of supposed independent variables may in fact be effects rather than causes.} \cite[~p. 35]{campbell} \\

\noindent \textit{But since we are interested in the way events unfold in the longer term, we want a set of empirical relationships that carry us deeper into the funnel and move outward from events and attitudes that are expressly political.} \cite[~p. 35]{campbell} \\

\noindent \textit{\textbf{Such laws we presume to exist, and with proper phrasing they should not only outlast reversals of voting patyters but should predict them.}} \cite[~p. 37]{campbell}

\subsection{Notas de aula}

\begin{description}
    \item [Sistema de Crenças:] basicamente, dizer que determinado elemento está associado a um outro a partir de explicações causais. Trata-se de mobilizar aspectos para responder ou classificar determinado grupo, o que corresponde a uma resposta \textbf{lógica} para a Escola de Michigan (ainda que eles não necessariamente gostem da resposta). 
    
    \textit{Configuração de ideias e atitudes em que os elementos estão ligados entre si por alguma forma de restrição ou interdependência funcional.} \cite[~p. 3]{converse_1964}. 
    
    Uma lista de características que são mobilizadas para descrever determinado grupo. É o que eu associo se quiser chamar alguém de conservador, liberal etc.
\end{description}

\noindent O eleitor \textbf{não} se guia por questões fundamentalmente racionais e, na prática, por questões de educação formal. Se você nem sequer sabe o que significa ser conservador, você não pode escolher seu voto de forma racional. Daí a irracionalidade do eleitor, que pauta seu voto por questões que não são atravessadas pela formalidade, mas por questões psicológicas. \\

\noindent O eleitor é racional quando consegue dar uma resposta mais sofisticada a respeito daquilo que ele, inclusive, não concorda. E é mais sofisticado e racional ainda quando consegue abstrair e explicar de forma abstrata o que determinado grupo político defende. O que incomoda os analistas que vieram a seguir é que o eleitor ideal da Escola de Michigan é o racional, que fez ensino superior e é intelectualmente sofisticado. \\

\noindent \textit{É claro que a ordenação dos indivíduos nesta escala vertical de informação se deve em grande parte a diferenças na educação, mas também é fortemente modificada por diferentes interesses e gostos especializados que os indivíduos adquiriram ao longo do tempo (um para a politica, outro para a atividade religiosa, outro para a pesca e assim por diante.} \cite[~p. 10]{converse_1964}. O que se fala justamente é ``olha como somos ilustrados por nos interessarmos por política! Política é um assunto que as pessoas simplesmente não se interessam e por isso o cidadão americano vota tão mal.". Dito de outra forma, assuntos da política nacional e internacional são de natureza abstrata ao homem comum. Segundo \citeonline[~p. 20]{converse_1964}, \textbf{o nível de informação das massas é baixo}.

$$ \text{O \ eleitor \ americano} = \text{O \ ``idiota" \ americano} $$

\noindent A literatura critica amplamente a Escola de Michigan, mas ninguém deixa de dialogar com ela --- é a Escola mais citada. \\

\noindent \textbf{Uma tipologia do eleitor americano:}

\begin{description}
    \item [Ideológico:] consegue responder questões sobre o espectro ideológico com facilidade e mobiliza descrições abstratas (eleitor ideal);
    \item [Quase-ideológico:] usam conceitos do campo ideológico, mas utilizam os termos de forma duvidosa;
    \item [Grupos de interesse:] avaliam partidos e candidatos em termos do potencial tratamento favorável ou desfavorável de diferentes grupos sociais;
    \item [Natureza dos tempos:] citava algo relacionado à esfera da política para classificar os partidos, mas não cabia nos estratos anteriores;
    \item [Sem conteúdo político:] avaliaram partidos e candidatos sem qualquer argumento político.
\end{description}

\subsection{Fórum}

\begin{description}
    \item [Pergunta:] Quais os argumentos que fazem Campbell e colegas, no livro The American Voter, a interpretar que o eleitor americano é irracional? Baseie a sua interação no fórum a partir da leitura do texto. 
\end{description}

Campbell et al. (1960), mais especificamente no capítulo Theoretical orientation, fazem uma proposição fundamentalmente metodológica para a análise do comportamento eleitoral do americano. Nessa empreitada, os autores apresentam suas percepções a respeito de como esse eleitor define seu voto. Daí a construção teórica do funil de causalidade: uma superestrutura que permite que o pesquisador determine, com certa precisão, quais variáveis foram responsáveis - ou, pelo menos, quais foram mais determinantes - na decisão final do eleitor ao computar seu voto nas urnas. E, mais do que isso, ao descobrir quais variáveis mais influenciam no comportamento, seria possível prever o voto e os resultados eleitorais de forma geral.  

Nesse caso, o sistema de crenças é um conceito importante. Trata-se, segundo Converse (2006), da capacidade de um indivíduo associar elementos a partir de explicações causais, servindo, no limite, para justificar o voto de forma racional. Se o eleitor ideal da Escola de Michigan é aquele capaz de explicar em termos abstratos aquilo que um grupo político defende, argumenta-se que o eleitor médio americano não é sequer capaz de mobilizar um sistema de crenças para decidir seu voto. 

Na prática, essa argumentação sugere de que modo os autores interpretam o voto do eleitor médio: não fruto da racionalidade, mas da conjugação de uma série de variáveis que levam com que o indivíduo se expresse politicamente de determinada forma. A irracionalidade na escolha do voto se justificaria pelos diferentes níveis educacionais dos eleitores americanos, fator que supostamente levaria a um natural desinteresse da população menos ilustrada pelo debate político. Finalmente, isso implicaria a decisão do voto a partir de questões meramente psicossociais, não pautadas pela racionalidade.

