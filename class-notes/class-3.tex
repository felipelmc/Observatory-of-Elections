\section{Os determinantes sociológicos do voto}

LAZARSFELD, P.; BERELSON, F.; GAUDET, B. Prefacio a la segunda edición (p. 1-23) e El
efecto de activación (p. 121-136). In: \_\_\_\_\_\_. El pueblo elige. \textit{Estudio del proceso de formación del voto durante una campaña presidencial}. Buenos Aires: Ediciones 3, 1962.

\subsection{Fichamento}
A conjuntura da Segunda Guerra Mundial teve enorme impacto nas ciências sociais, na medida em que o momento fazia necessário que sociólogos, psicólogos, antropólogos e economistas colocassem em prática seus conhecimentos teóricos para auxiliar nas tomadas de decisão durante o conflito. \\

\noindent \textit{El buen resultado que tuvieron estas investigaciones y los recomendaciones de ellas derivadas, han servido para acrecentar el prestigio de las ciencias sociales. La administración estatal, la industria y los grupos laborales se inclinan cada vez más a solicitar el asesoramiento de los especialistas en ciencias sociales. El curso de los acontecimientos de la posguerra ha dado aún mayor impulso a esta tendencia. La aparición de la bomba atómica nos ha hecho conscientes de que los descubrimientos de la
física han sobrepasado nuestra capacidad de integrarlos al sistema social que nos rige. Ante la probabilidad de una tercera guerra mundial, que contraría el deseo universal de paz, muchos se preguntan hasta qué punto los individuos que componen una sociedad pueden controlar los hechos sociales.} \cite[~p. 1]{lazarsfeld_berelson_gaudet} \\

\noindent Na prática, o início do texto chama atenção para a necessidade de uma articulação forte entre teoria e prática no campo das ciências sociais, considerando acontecimentos recentes (à época, evidentemente) da história mundial. Além disso, indica a tendência reprovável nas ciências sociais em realizar estudos e experimentos sem a profundidade necessária, desconsiderando ``\textit{La complejidad de la vida social}" \cite[~p. 2]{lazarsfeld_berelson_gaudet}. Por último, defende-se a necessidade de desenhos de pesquisa que busquem estudar o objetivo a partir das mudanças sociais ao longo do tempo. \\

\noindent \textbf{OBJETIVO DO ESTUDO:} estudar a formação, as mudanças e a evolução da opinião pública. Um grupo de especialistas entrevistou, durante 7 meses, 600 indivíduos uma vez ao mês no formato de painel, apesar de ter entrevistado outras pessoas. A ideia era compreender os efeitos da campanha presidencial na comunidade. \\

\noindent Um percentual de 13\% dos entrevistados mudou seu comportamento na eleição às vésperas. Daí, os autores formulam 3 perguntas fundamentais: (a) que classe de pessoas está mais predisposta ao \textit{cambio}?; (b) que influências atual para produzir esses cambios?; (c) em que direção se orientam os cambios? \\

\noindent A resposta para a primeira pergunta tem a ver com o que o autor chama de \textit{correlaciones externas} --- isto é, determinados grupos de fato estão mais predispostos a votar no partido A ao partido B por motivos variados. No entanto, essas correlações podem decepcionar. \\

\noindent O que explica a mudança de comportamento de 6 dos 65 indivíduos que pretendiam abster-se de votar? \textit{No fue difícil descubrir los factores que habían producido este cambio. El equipo encargado del estudio en el condado de Erie observó que, en esa oportunidad al menos, la organización proselitista republicana era mucho más activa y eficiente que la demócrata. Efectivamente, cuando se interrogó a los 6 mutantes mencionados acerca de la razón que los impulsó a asistir a las urnas, todos declararon que, a último momento, habían recibido la visita de un representante del Partido Republicano, quien los había convencido para que votaran.} \cite[~p. 4-5]{lazarsfeld_berelson_gaudet}. \\

\noindent O interessante do argumento é que, ao dividir em grupos menores que apresentam mudança de comportamento, o autor julga ser possível investigar quais são as motivações que influenciam nessa alteração. No caso desse estudo, o resultado foi que os contatos pessoais dos indivíduos foram os fatores que mais influenciaram na mudança de comportamento frente às urnas. Ainda é importante, na verdade, determinar em que direção ocorreu a mudança. E, nesse caso, o estudo indica: no que diz respeito ao âmbito dos subgrupos --- isto é, o comportamento dos indivíduos que participam de um mesmo subgrupo ---, as mudanças foram \textbf{homogêneas}. O mesmo não vale para o cenário geral, considerando o comportamento de todos o subgrupos, em que foi observada a tendência da \textbf{polarização}. Ou seja, se dentro de cada subgrupo a tendência dos indivíduos era pensar de forma semelhante entre si, o comportamento dos subgrupos de forma geral foi variada. \\

\noindent \textbf{Aliás, uma coisa que eu pensei: aqui ele fala em polarização porque a gente tá falando de Estados Unidos, ou seja, temos só Republicanos e Democratas com expressão eleitoral. Mas aqui no Brasil, como são/foram os estudos sobre isso? Porque aí a situação já era completamente diferente.} \\

\noindent Enuncia o cuidado necessário de que um determinado estudo tem seus resultados limitados pelo momento e pelo local em que foi realizado. Se a comparação entre dois estudos sugerir os mesmos resultados, chama-se \textit{función de corroboración}; depois, se ainda houver diferenças nos resultados estatísticos, mas a consideração das condições conduzir para a mesma conclusão geral, chama-se \textit{función de especificación}; e, por último, determinados resultados podem ser esclarecidos por estudos posteriores, e daí a \textit{función de eclarecimiento}. \\

\noindent O que se fez, então, foi a comparação com outro estudo. Através de características do perfil socioeconômico dos entrevistados (religião, zona de moradia), calculou-se um \textit{índice de predisposición política}, tornando possível determinar se o entrevistado votou em harmonia com o seu médio social ou se ele era um ``desviado", isto é, se a opção política estava fora do padrão esperado para ao subgrupo pertencente. \\

\noindent \textit{Al estudiar a los 54 mutantes de partido descubrimos que, antes de su cambio de opinión, 36 de estos sujetos tenían una intención electoral contraria a la predominante en su ambiente social, mientras que, después de producidos los cambios, sólo quedaron 20 casos `desviado'. De allí dedujimos que los cambios de partido se orientan en un sentido que conduce a una mayor armonía y homogeneidad dentro de cada subgrupo.} \cite[~p. 6]{lazarsfeld_berelson_gaudet}. \\

\noindent \textit{En efecto, el 38\% de los mutantes manifestaron que les era indiferente el resultado de los comicios; en cambio, entre los constantes, sólo un 21\% evidenció tal disposición. A esto se añade que el 65\% de los mutantes, en contraposición al 46\% de los constantes, juzgaban que no existía una verdadera diferencia entre los candidatos.} \cite[~p. 7]{lazarsfeld_berelson_gaudet}. Na prática, o indivíduo que muda de opinião não está realmente interessado no resultado final das eleições. Qualquer ocorrência do dia a dia pode interferir no seu voto. \\

\noindent Critica os cientistas sociais que consideram inútil a repetição de experimentos de um mesmo tipo de análise em situações idênticas e variadas. Na opinião dos autores, essa repetição é fundamental para que seja possível parar estudos e complementar os resultados. \\

\noindent Parei na página 8: Datos empíricos y procesos sociales.

\subsection{Notas de aula}

O eleitor é racional? Ora, o objetivo do texto não é discutir racionalidade do eleitor --- que, aliás, nem é discutida nessa teoria. Trata-se de um texto que discute metodologia. Precursor da Escola de Columbia. \\

\noindent Foco principal do livro é detalhamento do método de painel. Percepção da análise que associa o voto ao grupo ao qual o eleitor pertence. Aliás, foca nos mutantes em relação aos constantes. A preocupação não é explicar o resultado eleitoral, mas entender por que as pessoas mudam de voto através de determinantes sociológicos. \\

\noindent \textbf{Fatores determinantes em ordem de importância:}

\begin{enumerate}
    \item Núcleo familiar
    \item Outros contatos pessoais diretos (líderes de opinião)
    \item Campanha política
\end{enumerate}

\noindent Um ponto importante desse trabalho é que foi um dos primeiros que chamou a atenção para a influência dos formadores de opinião. A hipótese principal é que a influência vem da família, mas foram surpreendidos ao identificar que os formadores de opinião são um atalho informacional importante. Se alguém por quem você tem muito afeto, por exemplo, fez essa sugestão, você acompanha no voto. \\

\noindent Voto como tradição familiar: a grande maioria declarava que sabia em quem os pais e avós votaram e que isso influiu na escola do próprio voto. Não votar em outra pessoa é teimosia? Segundo o autor, não: o que você busca é gratificação pessoal --- eu quero ir para o almoço de domingo e ter paz. Além disso, sensação de coesão e pertencimento ao núcleo familiar. E finalmente, sensação de segurança pessoal (se sentir bem na sua família é um conforto extra. \\

\noindent \textbf{Viés de confirmação:} uma vez formada a opinião, você resiste à informação que a contraria. Nickerson (1998). \\

\noindent Segundo os dados, quanto maior o nível socio-econômico, maior o percentual de votos em republicanos em relação aos democratas. É o primeiro estudo sistemático a esse respeito. Metodologicamente, a determinação da classe de um indivíduo era feita pelas entrevistas: visual da casa, bairro, vestimenta, profissão e a partir daí deduzia-se a renda. Falho, evidentemente, mas era o possível para a época --- até porque as pessoas ainda estavam entendendo se responder questionários era normal, já que era algo novo. \textbf{IMPORTANTE: }Essa é uma crítica fundamental que a Escola de Michigan faz à Escola de Columbia: os aspectos metodológicos da pesquisa --- área restrita, dedução de renda possivelmente falha... \\

\noindent E qual é a vantagem metodológica de perguntar antes da eleição o quanto o eleitor nota de diferença entre os candidatos disputando a eleição? É pegar o eleitor no pulo: pensando em Bourdieu, os indivíduos tendem a pensar muito no que dizem. Ao perguntar com antecedência, evitamos que o eleitor dê uma resposta mais ``elaborada".