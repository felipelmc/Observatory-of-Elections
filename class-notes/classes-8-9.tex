\section{A teoria da escolha racional (I): voto prospectivo}

DOWNS, A. Introdução e Cap. 8 --- A estática e a dinâmica das ideologias partidárias. In: \_\_\_\_\_\_. \textit{Uma teoria econômica da democracia}. São Paulo: Editora da Universidade de São Paulo, 2013.

\subsection{Fichamento}

\noindent \textit{Não há razão a priori para presumir que esse ordenamento é racional, isto é, razoavelmente dirigido para a realização de objetivos conscientes. Entretanto, a teoria econômica se erigiu sobre a suposição de que prevalece a racionalidade consciente [...]. Já que nosso modelo ex definitione diz respeito ao comportamento racional, temos também que fazer essa suposição.} \cite[~p. 26]{downs} \\

\noindent \textit{A análise econômica, portanto, consiste em dois importantes passos: descoberta dos objetivos que aquele que toma decisão está perseguindo e análise de quais meios de atingi-los são mais razoáveis, isto é, exigem a menor aplicação de recursos escassos.} \cite[~p. 26]{downs} 

\begin{description}
    \item [Homem racional:] \textit{Mas a definição econômica se refere unicamente ao homem que se move em direção a suas metas de um modo que, ao que lhe é dado saber, usa o mínimo insumo possível de recursos escassos por unidade de produto valorizado.} \cite[~p. 27]{downs}. Na prática, não se trata de definir racionalidade a partir de proposições ``lógicas": você pode tomar decisões baseado em emoções e isso pode ser perfeitamente racional.
    
    \item [Homem racional (2):] \textit{[...] aquele que se comporta como se segue: (1) ele consegue sempre tomar uma decisão quando confrontado com uma gama de alternativas; (2) ele classifica todas as alternativas diante de si em ordem de preferência de tal modo que cada uma é ou preferida, indiferente ou inferior a cada uma das escolhas; (3) seu ranking de preferência é transitivo; (4) ele sempre escolhe, dentre todas as alternativas possíveis, aquela que fica em primeiro lugar em seu ranking de preferência; e (5) ele sempre toma a mesma decisão cada vez que é confrontado com as mesmas alternativas. Todos aqueles que tomam decisão racionalmente no nosso modelo --- inclusive partidos políticos, grupos de interesse e governos --- mostram as mesmas qualidades.} \cite[~p. 28]{downs}
\end{description}

\noindent \textit{[...] concentramos nossa atenção apenas nas metas econômicas e políticas de cada indivíduo ou grupo no modelo. Reconhecidamente, a separação dessas metas das muitas outras que os homens perseguem é bastante arbitrária. [...]. Todavia, esse é um estudo de racionalidade econômica e política, não de psicologia. Portanto, ainda que considerações psicológicas tenham um lugar legítimo e significativo tanto na economia quanto na ciência política, nós nos desviamos delas [...].} \cite[~p. 28-29]{downs} \\

\noindent \textit{A função política das eleições numa democracia, presumimos, é selecionar um governo. Portanto, comportamento racional vinculado às eleições é comportamento orientado para esse fim e nenhum outro.} \cite[~p. 29]{downs} \\

\noindent Fundamental diferenciar erros racionais de irracionalidade: \textit{Nosso desejo de desviar da irracionalidade política nasce de (1) a complexidade do assunto, (2) sua incompatibilidade com o nosso modelo de comportamento puramente racional e (3) o fato de ser um fenômeno empírico que não podemos tratar através apenas da lógica dedutiva mas que também exige real investigação para além do escopo deste estudo.} \cite[~p. 32]{downs} \\

\noindent \textit{Nosso modelo se baseia no pressuposto de que todo governo procura maximizar o apoio político. Presumimos ainda que o governo existe numa sociedade democrática em que se façam eleições periódicas, que seu objetivo principal é a reeleição, e que a eleição é o objetivo daqueles partidos agora alijados do poder. Em cada eleição, o partido que recebe o maior número de votos (embora não necessariamente a maioria) controla todo o governo até as próximas eleições, sem quaisquer votações intermediárias, seja pelo povo como um todo, seja pelo parlamento. O partido governante, portanto, tem liberdade ilimitada de ação, dentro dos limites da constituição.} \cite[~p. 33]{downs} \\

\textit{No nosso modelo, o governo persegue seu objetivo sob três condições: uma estrutura política democrática que permite a existência de partidos de oposição, uma atmosfera de graus variáveis de incerteza e um eleitorado de eleitores racionais.}

\textit{Nosso modelo mantém uma relação definida com modelos econômicos anteriores de governo, embora o nosso seja positivo e na maioria dos outros seja normativo. Buchanam propôs uma dicotomia entre concepções organísmicas e individualistas de Estado; tentamos evitar ambos os extremos. Samuelson e Buamol argumentaram que o Estado pode assumir eficientemente apenas transferências diretas de renda e ações que produzam benefícios indivisíveis; tentamos mostrar que tem muitos outros papéis legítimos. Bergson tentou estabelecer relações entre fins individuais e sociais por meio de um postulado puramente ético; adotamos um axioma ético sob forma política. Arrow provou que essas relações não poderiam ser estabelecidas racionalmente sem prescrição; tentamos mostrar como seu dilema pode ser contornado.}

\textit{Tentamos essas tarefas por meio de um modelo que é realista e, contudo, não preenche os detalhes das relações no interior dele. Em suma, queremos descobrir qual forma de comportamento político é racional tanto para o governo quanto para os cidadãos de uma democracia.} \cite[~p. 41-42]{downs} \\

\noindent \textit{Se as ideologias políticas são verdadeiramente meios para atingir a finalidade de obter votos, e se sabemos algo sobre a distribuição das preferências dos eleitores, podemos fazer previsões específicas a respeito de como as ideologias mudam em conteúdo à medida que os partidos manobram para conseguir o poder. Ou, inversamente, podemos afirmar as condições sob as quais as ideologias passam a se parecer umas com as outras, a divergir umas das outras, ou a permanecer em alguma relação fixa.} \cite[~p. 135]{downs} \\

\noindent \textbf{Sobre o voto prospectivo:} \textit{Mesmo num mundo certo, entretanto, a abstenção é racional para os eleitores extremistas que são orientados para o futuro. Estão dispostos a deixar o pior partido vencer hoje a fim de impedir que o partido melhor se movimente em direção ao centro, de modo que, em futuras eleições, ele esteja mais próximo deles. Assim, quando ele realmente vence, sua vitória é mais valiosa a seus olhos. A abstenção se torna, desse modo, uma ameaça a ser usada contra o partido mais próximo de nossa própria posição extrema, de modo a mantê-lo longe do centro.} \cite[~p. 140]{downs} \\

\noindent \textbf{Sobre sistemas políticos bimodais com eleitores distribuídos normalmente com modalidades próximas a cada extremo:} \textit{[...] qualquer um dos partidos que vença tentará implementar políticas radicalmente opostas à ideologia do outro partido, já que os dois estão em extremos opostos. Isso significa que a política governamental será altamente instável e que é provável que a democracia produza caos. Infelizmente, o crescimento de partidos de centro equilibradores é improvável.} \cite[~p. 141]{downs} \\

\noindent \textit{Em circunstâncias mais normais, em países onde há duas classes sociais opostas mas não há uma classe média bastante grande, é mais provável que a distribuição numérica se incline para a esquerda, com uma pequena modalidade na extrema direita.} \cite[~p. 142]{downs} \\

\noindent \textit{[...] fica claro que a distribuição numérica de eleitores ao longo da escala política determina, em grande medida, que tipo de democracia se desenvolverá.} \cite[~p. 142-143]{downs} \\

\noindent \textit{Como vimos no último capítulo, a integridade e a responsabilidade criam relativa imobilidade, o que impede que o partido dê saltos ideológicos sobre as cabeças de seus vizinhos. Desse modo, o movimento ideológico é restrito ao progresso horizontal no máximo até o --- e nunca além do --- partido mais próximo de cada lado.} \cite[~p. 143]{downs} \\

\noindent \textbf{Existe um limite para a introdução de novos partidos:} \textit{Os partidos existentes naquela altura se organizam por meio da competição, de modo que nenhum partido pode obter mais votos movimentando-se para a direita do que perde à esquerda fazendo o mesmo, e vice-versa. O sistema político atinge desse modo um equilíbrio de longo prazo no que diz respeito ao número e posições de seus partidos, presumindo-se que não houve qualquer mudança na distribuição dos eleitores ao longo da escala.} \cite[~p. 144]{downs} 

\begin{description}
    \item [Partido:] um grupo de homens que buscam chegar ao poder. Não sobrevive se nenhum dos membros se elege;
    \item [Sistema vencedor-leva-tudo:] tende a estreitar o campo a dois partidos concorrentes, porque o que ocorre na prática é uma amalgamação de partidos até que seja possível atingir o número mínimo de votos para eleger pelo menos um candidato;
    \item [Sistema proporcional:] estímulo ao multipartidarismo, porque o percentual de votos necessário para eleger um candidato é muito menor.
    \item [Pergunta:] a estrutura do sistema eleitoral determina a distribuição dos eleitores, ou a distribuição dos eleitores determina o sistema eleitoral?
\end{description} 

\textit{[...] é provável que [...] os partidos se empenhem em se distinguir ideologicamente uns dos outros e em manter a pureza de suas posições; ao passo que, em sistemas bipartidários, cada partido tentará se parecer com o seu oponente tanto quanto possível.} 

\textit{Se nosso raciocínio é correto, é provável que os eleitores em sistemas multipartidários oscilem muito mais em função de considerações doutrinárias --- questões de ideologia e políticas --- do que os eleitores em sistemas bipartidários. Esses últimos eleitores são aglomerados na amplitude moderada onde ambas as ideologias se situam; assim, é provável que considerem a personalidade, ou competência técnica, ou algum outro fator não-ideológico como decisivo. [...]} 

\textit{Os eleitores em sistemas multipartidários, entretanto, têm uma ampla gama de escolha ideológica, com os partidos antes enfatizando do que suavizando as diferenças doutrinárias. Desse modo, considerar as ideologias como fatores decisivos na nossa decisão de voto é geralmente mais racional num sistema multipartidário do que num sistema bipartidário. [...]} \cite[~p. 147-148]{downs} \\

\noindent Novos partidos podem contemplar dois objetivos: (1) \textit{ganhar eleições} ou (2) \textit{influenciar partidos já existentes}. \cite[~p. 148]{downs}. \textit{Fica claro que um pré-requisito importante para o aparecimento de novos partidos é uma mudança na distribuição de eleitores ao longo da escala política. [...]. Uma mudança no número de eleitores per se é irrelevante; é a distribuição que conta. Desse modo, o sufrágio feminino não cria quaisquer novos partidos, embora aumente o total de votos enormemente.} \cite[~p. 151-152]{downs}\\

\noindent \textit{Por um lado, quer agradar tantos eleitores quanto possível; por outro lado, quer ter um forte apelo para cada eleitor individualmente. O primeiro desejo implica uma plataforma que contém uma gama ampla de políticas que representam muitas perspectivas ideológicas diferentes. O segundo desejo implica uma integração íntima de políticas em torno de um ponto de vista filosófico de qualquer um dos eleitores que estiver sendo cortejado. Obviamente, quanto mais um desejo é alcançado, menos o outro será satisfeito.} \cite[~p. 153]{downs}\\

\noindent \textit{Portanto, o juízo que o eleitor faz de cada partido se torna bidimensional: ele deve contrabalançar a posição líquida do partido (a média de suas políticas) e seu arco (sua variação) ao decidir se quer apoiá-lo. Se um partido tem a média idêntica à posição do eleitor (que presumimos como sendo de valor único) mas uma variação enorme, ele poderá rejeitá-lo a favor de um outro partido com uma média não tão próxima a ele, mas com uma variação muito menor.} \cite[~p. 154]{downs}\\

\noindent \textbf{Sobre a ambiguidade de sistemas bipartidários:} \textit{A ambiguidade, portanto, aumenta o número de eleitores a quem um partido pode agradar. Esse fato encoraja os partidos, num sistema bipartidário, a serem tão ambíguos quanto possível em relação a suas posições sobre cada questão controversa. [...].}

\textit{[...] Naturalmente, isso faz com que se torne mais difícil para cada cidadão votar de modo racional; fica difícil para ele descobrir o que seu voto apóia quando dado a um ou a outro partido. Consequentemente, os eleitores são encorajados a tomar decisões com base em algo diferente das questões, isto é, com base na personalidade dos candidatos, em padrões tradicionais de voto familiar, na lealdade a antigos heróis partidários, etc. [...]. Somos obrigados a concluir que o comportamento racional por parte dos partidos tende a desencorajar o comportamento racional por parte dos eleitores.} \cite[~p. 157]{downs} \\

\noindent \textit{Um determinante básico do desenvolvimento político de uma nação é a distribuição de seus eleitores ao longo da escala política. Desse fator, em grande medida, depende se a nação terá dois ou muitos partidos importantes, se a democracia levará a um governo estável ou instável e se novos partidos substituirão continuamente os velhos ou desempenharão apenas um papel pequeno.} \cite[~p. 162]{downs}

\subsection{Notas de aula}

\noindent Não necessariamente nega o que veio antes, até porque escreve em um momento em que as Escolas Sociológica e Psicológica têm bastante influência (primeira edição do texto é de 1957). O objetivo é fazer uma análise complementar, inclusive porque a explicação do comportamento eleitoral é muito difícil. Esse texto é fundador da Escola de Teoria Racional do Voto. \\

\noindent Fundamentalmente, se o eleitor consegue ordenar preferências e consegue escolher entre A ou B, independentemente dos motivos parecerem racionais ou não ao pesquisador, então considero que ele é racional. Você assume que as pessoas têm preferências e precisa descobrir as preferências desses eleitores/partidos. Assim, você determina quais são os meios menos custosos para alcançar os objetivos, conseguindo então prever os resultados das eleições. \\

\noindent O importante é o meio, e não o fim. Por isso, as metas não precisam ser perenes. Além disso, as metas não precisam ser uma questão de livre escolha: decidir passar o dia atrás de comida é uma necessidade (e não uma escolha), mas mesmo assim isso não deixa de ser racional.

\begin{description}
    \item [O eleitor mediano:] \textit{``homo politicus" é o 'homem médio' do eleitorado, o 'cidadão racional' de nossa democracia modelo.} \cite[~p. 29]{downs}. 
\end{description}

\noindent O que conseguimos tentar prospectar são eleições majoritárias, quando há menos candidatos. \\

\noindent ``caroneiro": meu voto é só mais um e, portanto, não vou votar. \\

\noindent Passamos a lidar com o mundo empírico. Com isso, diminuímos muito as expectativas sobre o eleitor, assumindo a \textit{mea culpa} de que os seres humanos são imperfeitos e auto-interessados. \\

\noindent Por que chamamos de voto prospectivo? É você fazer um exercício de se posicionar em relação a determinadas pautas dos programas dos partidos. O eleitor, se conseguir identificar pautas programáticas de partidos diferentes, ele olha para o discurso e busca o partido mais próximo do que e acredita. Isto é, ele tenta prospectar um caminho futuro e identifica que está mais próximo do partido B do que do partido A. Nesse caso, o voto do eleitor seria \textbf{programático} (mas o partido não necessariamente é programático!!! Se moderar é interessante para ele). O eleitor aposta em um futuro a partir das posições político-partidárias disponíveis.

\subsection{Fórum}

\begin{description}
    \item [Pergunta:] Analise o comportamento do eleitorado brasileiro pela lente da teoria prospectiva do voto. Mobilize notícias de veículos de imprensa que detalham declarações dos eleitores para argumentar, baseado em Downs, como podemos compreender a decisão do voto na corrida presidencial de 2022.
\end{description}

\textit{Uma teoria econômica da democracia}, de Anthony Downs, apresenta uma abordagem que se diferencia das Escolas anteriormente estudadas --- ainda que não as negue --- ao defender que as decisões dos atores políticos (não só eleitores, mas também partidos e outros agentes) são racionais. Baseado em princípios da economia, a ideia fundamental é pensar que os atores lidam com recursos escassos e, por isso, têm um esforço racional de determinar o caminho menos custoso para atingir determinada meta. 

Em outras palavras: se um determinado ator é capaz de elencar preferências, ordená-las e escolher a que julgar mais favorável para si, então ele é racional. Pouco importam as motivações do eleitor para determinar qual é a melhor escolha em um determinado contexto, bastando que ele seja capaz de fazer esse exercício para que seja considerado racional. Ao supor que os atores possuem preferências, a descoberta delas permite que os pesquisadores tentem determinar os caminhos menos custosos para atingir determinado objetivo e, assim, prevejam resultados eleitorais. 

Uma série de assuntos vem aparecendo no debate público e frequentemente são mobilizados por políticos e eleitores. Em especial, a questão econômica: se por um lado o atual presidente celebra a distribuição do Auxílio Brasil, Lula o culpa pela crise econômica pela qual o Brasil vem passando. É possível supor que daí venha a maior preocupação dos eleitores de Lula com renda e emprego, conforme pesquisa Datafolha da semana passada. Outro caso diz respeito ao voto feminino: na pesquisa mais recente do Ipec, por exemplo, houve uma diminuição do percentual de mulheres que pretendiam votar no atual presidente, coincidindo com o ataque à jornalista Vera Magalhães no debate dos presidenciáveis. De forma geral, esses casos revelam a racionalidade dos eleitores em identificarem determinados aspectos da cena política e definirem seu comportamento nas urnas a partir disso. \\

