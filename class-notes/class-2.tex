\section{Introdução às Escolas}

\subsection{Escola de Michigan}
\begin{itemize}
    \item Behaviorismo --- análise da psicologia do voto;
    \item Escola diz que o eleitor é irracional; que o eleitor é burro e que ele faz escolhas a partir de incentivos irracionais;
    \item Teoria da identidade;
    \item Exemplo: você fala que é a favor da redução da desigualdade, política de igualdade de gênero, política de redistribuição de renda, e você é evangélica. Na hora do voto, apesar de todos esses valores, você escolhe o candidato evangélico mas que é contra todos esses valores que você acredita;
    \item Não só o fator religioso importa (apesar de ser o mais comum nesses estudos), mas existem outros elementos de identidade (ser do mesmo bairro, por exemplo);
    \item Principal explicação: falta de instrução;
    \item Gustave Le Bon – psicologia das multidões (no âmbito privado o indivíduo é mais racional, mas quando eu to em grupo minhas ações são levadas por emoção);
    \item Escola que começa a ganhar força em 1948 – período que nasce noz EUA a ANES (american national électoral studies) – instituto de pesquisa eleitoral (permitiu coletar e analisar dados eleitorais no âmbito nacional);
    \item ESEB – coletado pela Unicamp; principal fonte de análise das eleições brasileiras;
    \item Exemplos da influência da escola de Michigan: influencia as pesquisas eleitorais brasileiras até hoje; síndrome do Flamengo do Fábio Vanderley (eleitor brasileiro não é nada sofisticado pois ele trata política que nem ele trata futebol – uma vez que você cristaliza uma opinião, você não muda, e toda informação e evidência contrárias e desqualificação são revertidas para reforçar uma imagem positiva desse candidato favorito).
\end{itemize}

\subsection{Escola da Escolha Racional}
\begin{itemize}
    \item Faz crítica à escola de Michigan;
    \item O que determina seu voto? A economia;
    \item Eleitor é racional pois ele está reagindo aos incentivos externos que o direcionam para um determinado espectro político – ex: você tem dinheiro e vota num candidato que vai garantir a manutenção do seu status quo / ou não tem dinheiro e vai votar num candidato que promova uma mudança na sua realidade econômica;
    \item Mesmo sendo de outras ideologias, se o eleitor tá com dinheiro no bolso, o candidato é reeleito (exemplo: com a economia forte Lula e FHC foram reeleitos; Bolsonaro seria reeleito sem uma forte oposição);
    \item Fiorina (1978);
    \item Pessoas não são irracionais, elas acumulam conhecimento eleitoral ao longo do tempo (a partir das experiências eleitorais).
\end{itemize}

\noindent Ficou politicamente incorreto falar que um eleitor é irracional, mas os testes ainda são usados até hoje (como no ESEB) – ``ninguém deixou a escola de Michigan". \\

\noindent O teoria do voto retrospectivo também ainda tem muita força nas pesquisas contemporâneas. \\

\noindent Nova literatura sobre impacto da corrupção no comportamento do voto.