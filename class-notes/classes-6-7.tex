\section{A teoria da Cultura Política}

ALMOND, G. Public Opinion and National Security Policy. \textit{The Public Opinion Quarterly}, v. 20, n. 2, p. 371-378, Summer 1956. \\

\noindent ALMOND, G.; VERBA, S. Presentacion, Prologo e Un enfoque sobre la Cultura Politica. In: \_\_\_\_\_\_. \textit{La Cultura Civica}. Madrid, La Editorial Católica, 1970, p. 7-59. 


\subsection{Notas de aula}

\textbf{O antigo institucionalismo:}

\begin{itemize}

\item Análises especulativa / dedutiva;
\item Inspirados pela Filosofia Política e Direito;
\item Análises descritivas e formalistas;
\item Uma literatura normativa $\rightarrow$ as coisas ocorrem conforme as normas (constitucionais). Se está escrito nas normas, é o que vai acontecer na prática no mundo político;
\item Procurava ver a letra da lei e assumir, a partir das regras, como o mundo deveria ser;
\item Foco: estudar os princípios da Constituição e comparar se um país é mais ou menos governável a partir dela;

\end{itemize}

\noindent O \textbf{choque do institucionalismo} vem especialmente a partir do nazifascismo e da crise do liberalismo. Os pesquisadores dessa linha se percebem incapazes de explicar novos fenômenos políticos que surgiram a partir da década de 1930. Fundamentalmente, era de se esperar que o governo de Weimar fosse estável, por exemplo, mas na prática ocorreu o que ocorreu. \\

\noindent O \textbf{Behaviorismo} é pegar os incentivos da lei da física e da biologia e tentar entender o comportamento humano. Nesse caso, tentamos negar o institucionalismo a partir da suposição de que as regras não importam. Vamos buscar, assim, a cientificidade das ciências biológicas e exatas para compreender o comportamento humano, implementando esses preceitos na análise psicológica. \\

\noindent A formulação inicial vem da psicologia norte-americana. Recupera \textit{Behaviorism} de John Watson, escrito ainda em 1913. Perspectiva multidisciplinar: aproximação dos estudos políticos com esses métodos, teorias, pesquisas e resultados da psicologia, da sociologia, da antropologia e da economia. \\

\noindent A Escola de Michigan é o antro de onde são formados os grandes behaviorista. Daí o Modelo Michigan: 

\begin{itemize}
    \item Estudo sobre comportamento eleitoral baseado em um dos primeiros grandes \textit{surveys} eleitorais da história americana;
    \item A maioria dos eleitores é desinteressada e desinformada, votando segndo a identificação partidária muitas vezes herdada dos pais;
    \item O comportamento eleitoral mostrava-se fruto da identificação partidária.
\end{itemize}

\noindent \citeonline{almond_1970}\footnote{Primeira edição em 1963.} trabalha com um desenho de pesquisa extremamente caro para a década de 60. Um $n$ de cerca de 5.000 pessoas em 5 países (Inglaterra, Alemanha, Itália, México e Estados Unidos) divididas entre capitais, cidades menores e povoados (clivagem urbano $\times$ rural, testando a Teoria da Modernização\footnote{Note: se há um padrão de comportamento no comparativo entre aqueles que moram em regiões mais urbanizadas em relação à população rural, então a Teoria da Modernização ajudaria de alguma forma a explicar comportamento político.}). \textit{Queremos interpretar o que aconteceu na Itália e na Alemanha, mas por comparação --- pegando o protótipo de democracia (Inglaterra) e um lugar lindo, que é os Estados Unidos. Vamos pegar também um países ``em desenvolvimento, subdesenvolvido".} \\ 

\noindent Trata-se de um estudo sobre postura política do cidadão \textbf{de forma comparada}. Fundamentalmente, é a ideia de cientificidade por trás --- nenhuma escolha é feita por conveniência, mas, caso seja, ela deve ser detalhada para que o leitor/analista possa fazer sua crítica. Dito de outra forma: apresentar as escolhas e dar razões para essas escolhas. \\

\noindent \textit{O termo cultura política\footnote{\textbf{OBS.:} A Cultura Política da História não tem absolutamente nada a ver com a nossa Cultura Política.} refere-se especificamente a orientações políticas, posturas em relação ao sistema político e seus diferentes elementos, bem como atitudes em relação ao seu papel dentro desse sistema. Falamos de uma cultura política da mesma forma que podemos falar de uma cultura econômica ou religiosa.} \cite[~p. 30]{almond_1970}. \\

\noindent \textit{Nosso estudo decorre desse corpo teórico sobre as características e pré-condições da cultura da democracia. O que temos feito consiste em uma série de experimentos, a fim de testar algumas dessas hipóteses. Em vez de inferir as características de uma cultura democrática a partir de instituições políticas ou condições sociais, tentamos especificar seu conteúdo examinando atitudes em vários sistemas democráticos em funcionamento.} Somos \textbf{behavioristas}, e não institucionalistas! \\

\noindent \textit{Quando falamos da cultura política de uma sociedade, nos referimos à cultura política que informa os conhecimentos, sentimentos e valores de sua população. As pessoas são introduzidas em tal sistema, assim como são socializadas em papéis e sistemas sociais não políticos.} \cite[~p. 30]{almond_1970}. \\

\begin{description}
    \item [Argumento:] as pessoas têm um sistema de valores (sistema de crenças) que podem ser mais ou menos favoráveis à democracia (uma pessoa pode ter tendências mais totalitárias). 

    \item [Ideia de cultura política:] inclui conhecimentos, crenças, sentimentos e compromissos com valores políticos e com a realidade política. O conteúdo é resultado da socialização na infância, da educação, exposição aos meios de comunicação (informações que consome), experiências adultas com o governo, com a sociedade e com o desempenho econômico do país.
\end{description}

\noindent Tipos de orientação política:

\begin{description}
    \item [Orientação cognitiva:] capacidade de avaliar o mundo. Grau de conhecimento que os cidadãos têm do sistema político e o grau de crença no sistema, nos seus papéis e nos seus titulares, seus \textit{inputs} (o que você solicita como política pública) e \textit{outputs} (como essa demanda é atendida pelo governo);
    \item [Orientação afetiva:] você é irracional? Suas paixões/emoções te influenciam a ser mais democrático ou autoritário? Está relacionado com os sentimentos sobre o sistema política, seus papéis, pessoas e desempenho;
    \item [Orientação avaliativa:] São os julgamentos e opiniões sobre o sistema político que normalmente envolvem a combinação de critérios de valor com informações e sentimentos \cite[~p. 31]{almond_1970};
\end{description}

\noindent Análise sobre a visão do cidadão: as perguntas do survey assumem que o eleitor tem um conhecimento sobre o sistema político que não necessariamente corresponde à realidade. Isto é, há uma alta demanda por sofisticação intelectual --- influência da Escola de Michigan. \\

\noindent A partir das respostas, são delimitados 3 tipos de cultura cívica:

\begin{description}
    \item [Cultura paroquial:] : típica das estruturas políticas tradicionais. Ideia que pessoas não querem participar e não se interessam por política. Esse é o pior nível de cultura política pois você está alienado e apático à política. O modelo paroquial puro é o México;
    \item [Cultura súdita:] Congruente com estruturas políticas autoritárias e centralizadas. Existem países que tem uma cultura política onde as pessoas conhecem quem são seus governantes, seus sistemas, mas se submetem à condição de súditos. ``Não é uma cultura que queremos para nós". Caso da Itália e da Alemanha;
    \item [Cultura participante:] Condizente com uma estrutura política democrática. ``São os valores cívicos que queremos". Valores cívicos são a cultura política que pressupõe a participação. Caso dos EUA e Inglaterra. \\
\end{description}

\noindent Por que é um clássico? 

\begin{itemize}
    \item O primeiro que desenvolve a ideia de cultura cívica e cultura política --- Pioneiro ao investigar o papel desempenhado pela cultura política;
    \item Pioneiro em unir o estudo dos fundamentos psicológicos da política ao do sistema político. A proposta do trabalho não é observar comportamento eleitoral, e sim fazer um estudo comparativo;
    \item A partir disso, inaugurou as comparações entre os países no estudo da política --- estruturação da política comparada;
    \item Estimulou vários estudos posteriores, como o \textit{International Study on Values in Politics}, conduzido, no final dos anos 60, na Índia, Polônia, EUA e Iugoslávia. \\
\end{itemize}

\noindent Problemáticas da escola:

\begin{itemize}
    \item Determinismo cultural e etnocentrismo na definição do que deveria ser considerado como ``político";
    \item Crítica originada do Marxismo: não mostra as raízes históricas das culturas políticas das sociedades analisadas, frustrando uma interpretação concernente à classe e à estrutura;
    \item Abrem uma polêmica ao afirmar que o sucesso de um sistema política depende significativamente da compatibilidade entre suas instituições e conjunto de valores, crenças e atitudes partilhadas pela população (culturalistas VS neoinstitucionalistas) --- crítica mais forte.
\end{itemize}

\subsection{Fórum}

\begin{description}
    \item [Pergunta:] Qual o impacto da cultura política para o sistema político?
\end{description}

A idealização do conceito de cultura política é fruto de um processo mais amplo de reorientação teórica e metodológica nos estudos sobre regimes políticos. É uma tentativa de compreender determinados fenômenos históricos que a linha institucionalista foi incapaz de explicar ou prever, como a crise do liberalismo e a ascensão do nazifascismo. Se o Institucionalismo partia do pressuposto de que a letra da lei determinada pelas Constituições importava e era suficiente para a manutenção das democracias, o Behaviorismo recupera incentivos das ciências biológicas, exatas e da psicologia para compreender o comportamento humano.

Metodologicamente, torna-se importante a realização de pesquisas em larga escala e que levem em consideração a dimensão comparativa. Daí o desenho de pesquisa de Almond e Verba (1970), que conta com a entrevista de cerca de 5.000 indivíduos em 5 países diferentes: Inglaterra, Estados Unidos, Alemanha, Itália e México. 

Integrantes da Escola de Michigan, o argumento fundamental é de que os indivíduos possuem sistemas de crenças mais ou menos favoráveis ao estabelecimento de regimes democráticos. Isso é a cultura política: o resultado de uma série de processos sociais e psicológicos (como a socialização, níveis educacionais, ambiente familiar etc) que conformam os valores, os sentimentos e a postura dos indivíduos em relação ao sistema político.

A análise dos dados coletados na pesquisa possibilitou a descrição de 3 tipos de cultura: a paroquial, a súdita e a participante. No primeiro, os indivíduos são alienados e apáticos em relação à política; no segundo, os indivíduos se submetem à condição de súditos frente aos governantes; e, por último, o tipo mais propício ao estabelecimento de um regime democrático. Segundo os autores, então, o tipo de sistema político que pode ser implementado em um país é determinado, ou pelo menos significativamente dependente, do nível de cultura política dos cidadãos.
